\mychapter{Introdução aos sistemas de tempo real}
\section{Visão geral}
A utilização de sistemas de tempo real cresce cada vez mais e, em um mercado
competitivo, qualquer melhora de desempenho implica em maiores possibilidades de
venda. Este capítulo introdutório tentará mostrar os principais conceitos
relacionados a esse tipo de sistema, mostrando algumas das principais
dificuldades encontradas pelos programadores ao longo do desenvolvimento de
aplicações de tempo real.

\section{Desenvolvimento de sistemas de tempo real}
A integridade de muitos sistemas e dispositivos na sociedade moderna depende não
só dos efeitos ou resultados que esses produzem, mas também do tempo em que
esses resultados são produzidos. Os sistemas de tempo real, os quais estão
diretamente relacionados com essa característica, abrangem então desde de
sistemas de ``freio abs'' até o monitoramento de sinais vitais em um hospital.

Ao contrário dos sistemas de \soft\ convencionais, os sistemas de tempo real
estão intrinsecamente relacionados com o ambiente em que estão operando.
Exemplos desses tipos de sistemas podem ser encontrados no cotidiano: sistemas
de anti-travamento de freios computadorizados, monitoramento de sinais vitais,
aviação, estações espaciais, satélites geoestacionários, controladores e
manipuladores robóticos, controle de plantas e processos industriais,
ferramentas multimídia, sistemas de realidade virtual, comunicação sem fio,
telescópio astronômicos, dentre outros. Sabendo disso, pode-se perguntar: o que
seria então um sistema de tempo real?  Por que tal termo é utilizado? 

De maneira simplificada, um sistema computacional é dito {\it Sistema de Tempo
Real} (STR) quando a integridade dos dados produzidos dependem não só do
comportamento e da manipulação lógica dos dados, mas também do instante em que
esses dados são produzidos. Normalmente, sistemas desse tipo reagem aos sinais
de entrada de maneira dinâmica, mudando seu estado de configuração de acordo o
processo que está sendo monitorado.

Em sistemas críticos, tais como em controle de tráfego aéreo e ferroviário, é
comum fazer uso de múltiplas {\it threads}, múltiplos processos ou ainda
de processamento distribuídos para que se alcance o desempenho requerido para
esse tipo de aplicações. Por esse motivo, termos como {\it multi-thread} e {\it
multi-task} são algumas vezes utilizados, de maneira equivocada, como sinônimos
de sistemas de tempo real.

\section{Características dos STR}
Existem muitas características que diferem STR dos sistemas computacionais
convencionais. Entretanto, deve-se destacar que nem todas as características se
aplicam a qualquer tipo de STR. Algumas dessas características são:

\begin{itemize}
    \item Tempo de resposta limitado
    \item Escalonamento de processos
    \item Programação em baixo nível
    \item \Soft\ fortemente ligado a um {\it hardware} específico
    \item Funções especializadas
    \item Variáveis voláteis
    \item Implementação multi-tarefa
    \item Ambientes de execução imprevisíveis
    \item Execução contínua do sistema
    \item Aplicações críticas
\end{itemize}

\subsection{Aspectos importantes}
Sabe-se que todo STR está diretamente relacionado com as restrições temporais de
operação. Uma forma bem comum dessas restrições são os {\it deadlines}
estabelecidos para a conclusão das atividades. Um {\it deadline} especifica o
tempo máximo permitido para que uma tarefa seja concluída e seus resultados
estejam disponíveis para serem acessados pelo usuário.

Devido a essa característica, os STR são em sua maioria implementados em uma
linguagem que permitem que o programador acesse os recursos disponíveis de
maneira rápida e eficiente. Linguagens como C e C++, comumente utilizadas para o
desenvolvimento de sistemas operacionais, são muitas vezes utilizadas por
permitir isso. 

Contudo, há ainda aplicações em que nem mesmo essas linguagens serão utilizadas,
sendo necessário que todo o sistema, ou boa parte deste, seja implementado em
linguagens de baixo nível. Em muitos dos casos, recursos blocos {\tt asm} em C
serão utilizados para estabelecer um limiar entre o desenvolvimento em alto
nível e em baixo nível.

\section{Exemplos de sistemas de tempo real}
