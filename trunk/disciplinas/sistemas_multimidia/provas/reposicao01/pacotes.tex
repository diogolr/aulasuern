% Pacotes Principais -----------------------------------------------------------
\usepackage[portuges,brazil]{babel}
\usepackage[utf8]{inputenc}

\usepackage[left=3cm,right=2cm,top=3cm,bottom=2cm]{geometry}

% Figuras e Imagens ------------------------------------------------------------
\usepackage{graphicx}
% Figuras lado a lado
\usepackage{epsfig}
\usepackage{subfigure}

% Utilizar H para inserir as imagens REALMENTE onde eu desejo
\usepackage{float}

% Fontes -----------------------------------------------------------------------
\usepackage[T1]{fontenc}
\usepackage{pslatex}

% Simbolos ---------------------------------------------------------------------
\usepackage{textcomp}
\usepackage{amsmath}

% Setas extensas
\usepackage{extarrows}

% Cancelamento
\usepackage{cancel}

% Tabelas ----------------------------------------------------------------------
%\usepackage{multicol}
\usepackage{multirow}
% Colorir a tabela
\usepackage{colortbl}

% Outros pacotes ---------------------------------------------------------------
\usepackage{noitemsep}

\usepackage{color}
\usepackage{xcolor}

% Comentários em bloco
\usepackage{verbatim}

% Sublinhado, traçado...
\usepackage{ulem}

% Desenhar
\usepackage{tikz}
