% Comandos gerais --------------------------------------------------------------
\newcommand{\titulo}{Sistema de Inferência Neural e Processamento Estatístico
                     Multivariável Aplicado a Indústria do Petróleo}
\newcommand{\autor}{Diogo Leite Rebouças}
\newcommand{\emailautor}{\tt diogolr@dca.ufrn.br}
\newcommand{\dataaprovacao}{10 de junho de 2009}

% Configuração da fonte
\renewcommand{\familydefault}{\sfdefault}

% Comandos matemáticos ---------------------------------------------------------
% Implicação em fórmulas
\newcommand{\implica}{$\quad\Rightarrow\quad$} %Meio de linha
\newcommand{\implicafim}{$\quad\Rightarrow$}   %Fim de linha
\newcommand{\tende}{$\rightarrow$}

% Fração com parenteses
\newcommand{\pfrac}[2]{\parent{\frac{#1}{#2}}}

% Transformada de Laplace e transformada Z
\newcommand{\lapl}{\pounds}
\newcommand{\transfz}{\mathcal{Z}}

% Sequências
\newcommand{\sequencia}[4]{$#1_{#2}$, $#1_{#3}$, \ldots, $#1_{#4}$}

% Outros ----------------------------------------------------------------------
\newcommand{\chave}[1]{\left\{#1\right\}}
\newcommand{\colchete}[1]{\left[#1\right]}
\newcommand{\parent}[1]{\left(#1\right)}

\let\D\displaymath

\newtheorem{definicao}{Definição}
\newtheorem{exemplo}{Exemplo}
\newtheorem{lema}{Lema}
\newtheorem{observ}{Observação}
\newtheorem{teorema}{Teorema}

\newcommand{\defin}[1]{\begin{definicao}#1\end{definicao}}
\newcommand{\exemp}[1]{\begin{exemplo}#1\end{exemplo}}
\newcommand{\obs}[1]{\begin{observ}#1\end{observ}}

\newcommand{\pixel}{{\it pixel}}

% Transparência dos tópicos
\setbeamercovered{transparent}

% Sumário entre as subseções
%\AtBeginSubsection[]
%{
%\begin{frame}
%\frametitle{Sumário}
%\tableofcontents[currentsection,currentsubsection]
%\end{frame}
%}
